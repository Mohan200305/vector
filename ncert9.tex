\documentclass{article}
\usepackage{amsmath}
\usepackage{xcolor}
\usepackage{gensymb}
\usepackage{ragged2e}
\usepackage{graphicx}
\usepackage{gensymb}
\usepackage{mathtools}
\newcommand{\mydet}[1]{\ensuremath{\begin{vmatrix}#1\end{vmatrix}}}
\providecommand{\brak}[1]{\ensuremath{\left(#1\right)}}
\providecommand{\norm}[1]{\left\lVert#1\right\rVert}
\newcommand{\solution}{\noindent \textbf{Solution: }}
\newcommand{\myvec}[1]{\ensuremath{\begin{pmatrix}#1\end{pmatrix}}}
\let\vec\mathbf
\begin{document}
\begin{center}
        \textbf\large{CHAPTER-9 \\ TRIANGLES}
\end{center}
\section{Exercise 11.2}
Question(4).Construct a triangle $XYZ$ in which $\angle{Y}=30\degree$,$\angle{Z}=90\degree$ and $XY+YZ+ZX=11cm$. \\
\textbf{Solution:}\\
Let $\vec{X}$,$\vec{Y}$ and $\vec{Z}$ are the vertices of the triangle with coordinates.
Given $XY+YZ+ZX=8cm$.So the coordinate of the vertice  $\vec{X}$ is:
\begin{align}
{
\vec{X} =\myvec{0\\0}
}
\end{align}
Also given $\angle{\vec{Y}}=30\degree$ and $\angle{\vec{Z}}=90\degree$ so by finding the length of sides we can form a required triangle. \\
 The input parameters for this construction are\\
 \begin{table}[h]
	 \centering
	  \begin{tabular}{|c|c|c|} 
  \hline 
  \textbf{Symbol}&\textbf{Value}&\textbf{Description}\\ 
  \hline 
  $c+a+b$ & 11 & $XY+YZ+ZX$ \\ 
  \hline 
 $\angle{Y}$ & 30$\degree{}$ & $\angle{Y}$ in $\triangle$$ABC$\\ 
  \hline 
        $\angle{Z}$ & 90$\degree{}$ & $\angle{Z}$ in $\triangle$$XYZ$ \\
   
  \hline  
 $\vec{e_1}$ & $\myvec{ 
   1 \\
   0 \\
   0 
   }$ & Basis vector\\ 
 \hline
 \end{tabular}\\	

	 \caption{Parameters}
	 \label{tab:table1}
 \end{table}\\
From the given information\\
 \begin{align}
     a+b+c=k\\
	 b\cos{\vec{Z}}+c\cos{\vec{Y}}-a=0\\
	 b\sin{\vec{Z}}-c\sin{\vec{Y}}=0
 \end{align}
 Resulting in the matrix equations:
 \begin{align}
	 \myvec{1 & 1 & 1\\-1 & \cos{\vec{Z}} & \cos{\vec{Y}}\\0 & \sin{\vec{Z}} & -\sin{\vec{Y}}}\myvec{a \\ b \\ c}=k\vec{e_1}
 \end{align}
 Substituting the values of $k$,$\vec{e_1}$,$\angle{\vec{Y}}$,$\angle{\vec{Z}}$
 \begin{align}
     \myvec{1 & 1 & 1\\-1 & \cos{90\degree} & \cos{30\degree}\\ 0 & \sin{90\degree} & -\sin{30\degree}}\myvec{a \\ b \\ c}= 11\myvec{1 \\ 0 \\ 0 }
 \end{align}
  \begin{align}
	  \myvec{1 & 1 & 1\\-1 & 0 & \sqrt{3}/2\\0 &  1 & -1/2}\myvec{a \\ b \\c}= \myvec{11 \\ 0 \\ 0 }
  \end{align}
 Using row reduction methods to bring the values of $a$,$b$,$c$ into row-reduced echelon form,
 \begin{align}
     \xrightarrow[]{R_2 \rightarrow R_2+R_1}\myvec{1 & 1 & 1\\ 0 & 1 & \frac{\sqrt{3}}{2}+1\\0 &  1 & \frac{-1}{2}}\myvec{a \\ b \\ c}= \myvec{11 \\ 11 \\ 0 }\\
     \xrightarrow[]{R_1 \rightarrow R_1-R_2}\myvec{1 & 0 & \frac{-\sqrt{3}}{2}\\0 & 1 & \frac{\sqrt{3}}{2}+1\\0 &  0 & \frac{-1}{2}}\myvec{a \\ b \\ c}= \myvec{0 \\ 11 \\ 0 }\\
     \xrightarrow[]{R_3 \rightarrow R_3-R_2}\myvec{1 & 0 & \frac{-\sqrt{3}}{2}\\0 & 1 & \frac{\sqrt{3}}{2}+1\\0 &  0 & \frac{-\sqrt{3}-3}{2}}\myvec{a \\ b \\ c}= \myvec{0 \\ 11 \\ -11 }\\
     \xrightarrow[]{R_3 \rightarrow \frac{2}{-\sqrt{3}-3} R_3}\myvec{1 & 0 & \frac{-\sqrt{3}}{2}\\0 & 1 & \frac{\sqrt{3}}{2}+1\\0 &  0 & 1}\myvec{a \\ b \\ c}= \myvec{0 \\ 11 \\ \frac{22}{3+\sqrt{3}} }\\
     \xrightarrow[]{R_1 \rightarrow R_1+\frac{\sqrt{3}}{2}R_2}\myvec{1 & 0 & 0 \\ 0 & 1 & \frac{\sqrt{3}}{2}+1\\0 &  0 & 1}\myvec{a \\ b \\ c}= \myvec{\frac{11\sqrt{3}}{3+\sqrt{3}} \\ 11 \\\frac{22}{3+\sqrt{3}}  }\\
	 \xrightarrow[]{R_2 \rightarrow R_2-(\frac{\sqrt{3}}{2}+1)R_3}\myvec{1 & 0 & 0 \\ 0 & 1 & 0\\0 &  0 & 1}\myvec{a \\ b \\ c}= \myvec{\frac{11\sqrt{3}}{3+\sqrt{3}} \\ 11(1-\frac{\sqrt{3}+2}{\sqrt{3}+3}) \\\frac{22}{3+\sqrt{3}}}
 \end{align}
 After reduction the values of $a$,$b$,$c$ are:
 \begin{align}
     \myvec{a \\ b \\ c}=\myvec{\frac{11\sqrt{3}}{3+\sqrt{3}} \\ 11(1-\frac{\sqrt{3}+2}{\sqrt{3}+3}) \\\frac{22}{3+\sqrt{3}}}
 \end{align}
 Therefore the coordinates of the vertices are:
 \begin{align}
      \vec{X}=\myvec{0\\0}\\
      \vec{Z}=\myvec{b \\ 0}=\myvec{11(1-\frac{\sqrt{3}+2}{\sqrt{3}+3}) \\ 0}\\                                               
      \vec{Y}=\myvec{b \\ a}=\myvec{11(1-\frac{\sqrt{3}+2}{\sqrt{3}+3}) \\ \frac{11\sqrt{3}}{3+\sqrt{3}} }
 \end{align}
 Construction:\\
 \begin{figure}[h]
	 \begin{center}
		 \includegraphics[width=\columnwidth]{code(m)/fig.pdf}
	 \end{center}
	 \caption{Triangle XYZ}
	 \label{fig:Fig1}
 \end{figure}

\end{document}
